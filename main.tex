\\documentclass[conference]{IEEEtran}

\IEEEoverridecommandlockouts
\usepackage{cite}
\usepackage{amsmath,amssymb,amsfonts}
\usepackage{algorithmic}
\usepackage{graphicx}
\usepackage{textcomp}
\usepackage{xcolor}
\def\BibTeX{{\rm B\kern-.05em{\sc i\kern-.025em b}\kern-.08em
    T\kern-.1667em\lower.7ex\hbox{E}\kern-.125emX}}
\begin{document}

\title{Omnetpp dan Algoritma Dijkstra}

\author{\IEEEauthorblockN{Fajar Bimantara}
\IEEEauthorblockA{\textit{Institut Teknologi Batam} \\
\textit{Teknik Komputer}\\
Batam, Indonesia \\
1922027@student.iteba.ac.id}
\and

\IEEEauthorblockN{Muhammad Riyadhtul Akbar}
\IEEEauthorblockA{\textit{Institut Teknologi Batam} \\
\textit{Teknik Komputer}\\
Batam, Indonesia \\
1922011@student.iteba.ac.id}
\and

\begin{document}
\maketitle

\begin{abstract}
Jaringan wireless LAN adalah jaringan yang mengkoneksikan dua komputer atau lebih menggunakan
sinyal radio, cocok untuk berbagi-pakai file, printer, atau akses internet. Dengan mempergunakan
perangkat radio maka akan dapat membuat LAN tanpa menggunakan kabel data yang umum dipakai dalam
sebuah jaringan komputer. Setiap PC pada jaringan wireless dilengkapi dengan sebuah radio tranceiver atau
biasanya disebut adapter atau kartu Wireless LAN yang akan mengirim dan menerima sinyal radio dari dan
ke PC lain dalam jaringan. Teknologi wireless memungkinkan jaringan ini dapat dipasang di tempat
dimana jaringan kabel tidak dapat dipasang.
\end{abstract}

\begin{IEEEkeywords}
    wireless lan, network
\end{IEEEkeywords}

\section{Introduction}
Jaringan wireless LAN adalah jaringan yang mengkoneksikan dua komputer atau lebih menggunakan sinyal radio, cocok untuk berbagipakai file, printer, atau akses internet. Bila user ingin mengkoneksikan dua komputer atau lebih di lokasi yang sulit atau tidak mungkin untuk memasang kabel jaringan, sebuah jaringan wireless (tanpa kabel) mungkin cocok untuk
diterapkan.
Jaringan komunikasi wireless memberikan kemudahan dan fleksibilitas yang tinggi bagi
para pemakainya untuk dapat mengadakan hubungan komunikasi dengan sesama pemakai jaringan wireless maupun dengan pemakai lain yang terhubung dengan jaringan yang memakai media transmisi kabel (wired network). Wireless LAN (WLAN) menyediakan suatu alternatif bagi LAN tradisional berbasis twisted pair, kabel
koaksial, dan serat optik Wireless LAN mengirim dan menerima data melalui udara, dan meminimalkan penggunaan sambungan kabel. Jadi, Wireless LAN memiliki fleksibelitas, mendukung mobilitas, memiliki
teknik frequency reuse, selular dan handover, menawarkan efisiensi dalam waktu (penginstalan) dan biaya (pemeliharaan dan penginstalan ulang di tempat lain), mengurangi pemakaian kabel dan penambahan jumlah
pengguna dapat dilakukan dengan mudah dan
cepat.

\section{Komponen Wireless Lan}
Komponen utama dalam membangun
sebuah jaringan Wireless LAN adalah : 
-Access Point 
Merupakan perangkat yang menjadi sentral koneksi dari klien ke ISP. Berfungsi mengkonversikan sinyal frekuensi radio (RF) menjadi sinyal digital yang akan disalurkan melalui kabel, atau disalurkan ke perangkat WLAN yang lain dengan dikonversikan ulang menjadi sinyal frekuensi radio

-Wireless LAN Interface
Merupakan device yang dipasang di AccessPoint atau Mobile/Desktop PC, device yang dikembangkan secara massal adalah dalam bentuk PCMCIA (Personal Computer Memory Card International Association) card.

-Wired LAN
Merupakan jaringan kabel yang sudah ada, jika wired LAN tidak ada maka hanya sesama WLAN saling terkoneksi.

-Mobile/Desktop PC
Merupakan perangkat keras untuk klien, mobile PC pada umumnya sudah terpasang port PCMCIA sedangkan desktop PC harus ditambahkan PC card PCMCIA dalam bentuk ISA (Industry Standard Architecture) atau PCI
(Peripheral Component Interconnect) card.

-Topologi Wireless Lan
Wireless LAN memungkinkan dua bentuk koneksi, yang dikenal sebagai Ad-Hoc dan mode Infrastructure.

-Mode Ad-Hoc
Mode Ad-Hoc adalah suatu kondisi jaringan wireless yang tidak menggunakan access point.Artinya, antar client langsung terkoneksi satu dengan yang lainnya. Jika merasa asing dengan istilah Ad-Hoc, mungkin istilah Peer-to-peer dapat lebih mempermudah mengenali koneksi Ad-Hoc. Prinsip kerjanya sama saja dengan
Peer-to-peer. Disini setiap client akan saling terkoneksi secara langsung.


\section{Mode Infrastructure}
Model infrastructure adalah kondisi suatu jaringan dengan menggunakan suatu titik pusat yaitu access point. Semua client terhubung ke jaringan harus terkoneksi ke access point terlebih dahulu, baru kemudian dapat mengakses resource dari network/client lain yang ada. Untuk topologi infrastruktur, tiap PC mengirim dan menerima data dari sebuah titik akses, yang dipasang di dinding atau langit-langit berupa
sebuah kotak kecil berantena. Saat titik akses menerima data, ia akan mengirimkan kembali
sinyal radio tersebut (dengan jangkauan yang lebih jauh) ke PC yang berada di area cakupannya, atau dapat mentransfer data melalui jaringan Ethernet kabel. Titik akses pada sebuah jaringan infrastruktur memiliki area cakupan yang lebih besar.

\section{Cara Kerja Wireless Lan}
Setiap PC pada jaringan wireless dilengkapi dengan sebuah radio tranceiver, atau biasanya disebut adapter atau kartu Wireless LAN, yang akan mengirim dan menerima sinyal radio dari dan ke PC lain dalam jaringan.
Mirip dengan jaringan Ethernet kabel, sebuah Wireless LAN mengirim data dalam bentuk paket. Setiap adapter memiliki nomor ID yang permanen dan unik yang berfungsi sebagai sebuah alamat, dan tiap paket selain berisi data juga menyertakan alamat penerima dan pengirim paket tersebut. Sama dengan sebuah adapter Ethernet, sebuah kartu Wireless LAN akan memeriksa kondisi jaringan sebelum mengirim paket ke dalamnya. Bila jaringan dalam keadaan kosong, maka paket langsung dikirimkan.

\section{Kesimpulan}
Sistem Wireless LAN memberi kemudahan bagi user untuk mengakses informasi realtime dimanapun mereka berada. Faktor mobilitas ini mendukung produktifitas dan kesempatan layanan yang tidak mungkin dilakukan dengan jaringan kabe. Kecepatan dan kemudahan pemasangan / instalasi Wireless LAN dapat dipasang dengan cepat dan mudah, dan dapat membatasi keperluan pemasangan kabel melalui dinding dan langit-langit (plafon).

\bibliographystyle{IEEtran}
http://www.windowsnetworking.com
Jonathan L, Pejman R,802.11 Wireless
Lan Fundamentals, Cisco Press, 2007
Spencer M., Build Your Own Wireless
Lan, McGraw-Hill, 2002
\bibliography{referensi.bib}

\end{document}
